%%
%% Modified by Ricardo Garcia-Rosas to satisfy the rules established by the University of Melbourne's Research Higher Degrees Committee as of 4th of June 2019.
%% Guidelines can be found at: https://gradresearch.unimelb.edu.au/__data/assets/pdf_file/0004/2027839/Preparation-of-GR-theses-rules.pdf
%%
%%%%%%%%%%%%%%%%%%%%%%%%%%%%%%%%%%%%%%%%%%%%%%%%%%%%%%%%%%%%%%%%%%%%%%%%%
%% IMPORTANT NOTE TO AUTHOR:
%% As part of the guidelines, the use of the university logo is not permitted. This template contains it to make it easier to find/recognise in the Overleaf Gallery. To make the template compliant please go to 'Thesis.cls' and comment out the \includegraphics command in line 217 (it is clearly highlited).
%%%%%%%%%%%%%%%%%%%%%%%%%%%%%%%%%%%%%%%%%%%%%%%%%%%%%%%%%%%%%%%%%%%%%%%%%
%%
%% ----------------------------------------------------------------
%% Thesis.tex -- MAIN FILE (the one that you compile with LaTeX)
%% ---------------------------------------------------------------- 

% Set up the document
\documentclass[a4paper, 12pt, oneside]{Thesis}  % Use the "Thesis" style, based on the ECS Thesis style by Steve Gunn
%
% Put your figures in this directory
\graphicspath{Figures/}  % Location of the graphics files (set up for graphics to be in PDF format)
%

% Include any extra LaTeX packages required
\usepackage[square, numbers, comma, sort&compress]{natbib}  % Use the "Natbib" style for the references in the Bibliography
\usepackage{verbatim}  % Needed for the "comment" environment to make LaTeX comments
\usepackage{vector}  % Allows "\bvec{}" and "\buvec{}" for "blackboard" style bold vectors in maths
\hypersetup{urlcolor=blue, colorlinks=true}  % Colours hyperlinks in blue, but this can be distracting if there are many links.

%%%%%%%%%%%%%% my personal preamble %%%%%%%%%%%%%%%%
\usepackage[T1]{fontenc}
\usepackage[utf8]{inputenc}
\usepackage{graphicx,setspace,float,fancyhdr,hyperref,quiver,bbold}
\usepackage{amsmath,mathtools,amssymb,amsthm,cancel,mathdots,tikz-cd,xfrac,xcolor,amsfonts,esint}
\usepackage[all]{xy}
\usepackage[mathscr]{euscript}
\tikzcdset{scale cd/.style={every label/.append style={scale=#1},
    cells={nodes={scale=#1}}}}
% \setlength{\parindent}{0in}
\makeatletter
\DeclareMathOperator{\oeq}{\mathbin{\mathpalette\make@circled=}}%get your circled symbols here!
\DeclareMathOperator{\oneq}{\mathbin{\mathpalette\make@circled\neq}}
\newcommand{\make@circled}[2]{%
  \ooalign{$\m@th#1\smallbigcirc{#1}$\cr\hidewidth$\m@th#1#2$\hidewidth\cr}%
}
\newcommand{\smallbigcirc}[1]{%
  \vcenter{\hbox{\scalebox{0.77778}{$\m@th#1\bigcirc$}}}%
}
\makeatother
\renewcommand{\Re}{\mathrm{Re}}
\renewcommand{\Im}{\mathrm{Im}}
\newcommand{\bA}{\mathbb{A}}
\newcommand{\bC}{\mathbb{C}}
\newcommand{\bF}{\mathbb{F}}
\newcommand{\bN}{\mathbb{N}}
\newcommand{\bP}{\mathbb{P}}
\newcommand{\bQ}{\mathbb{Q}}
\newcommand{\bR}{\mathbb{R}}
\newcommand{\bZ}{\mathbb{Z}}
\newcommand{\lt}{\left}
\newcommand{\rt}{\right}
\newcommand{\rgl}{\rangle}
\newcommand{\lgl}{\langle}
\newcommand{\PaB}{\mathbf{PaB}}
\newcommand{\PaBhat}{\widehat{\mathbf{PaB}}}
\newcommand{\GT}{\mathbf{GT}}
\newcommand{\GThat}{\widehat{\mathbf{GT}}}
%change the style of arrows
\newcommand{\ra}{\longrightarrow}
\newcommand{\lra}{\longleftrightarrow}
\newcommand{\xra}{\xrightarrow}
\newcommand{\Ra}{\Rightarrow}
\newcommand{\Lra}{\Leftrightarrow}
\newcommand{\e}{\varepsilon}
%\det, \dim, \ker, \lcm, \gcd, all trig functions, already built-in
\DeclareMathOperator{\id}{\mathrm{id}}
\DeclareMathOperator{\im}{\mathrm{im}}
\DeclareMathOperator{\tr}{\mathrm{tr}}
\DeclareMathOperator{\proj}{\mathrm{proj}}
\DeclareMathOperator{\End}{\mathrm{End}}
\DeclareMathOperator{\Hom}{\mathrm{Hom}}
\DeclareMathOperator{\Ext}{\mathrm{Ext}}
\DeclareMathOperator{\Tor}{\mathrm{Tor}}
\DeclareMathOperator{\Span}{\mathrm{Span}}
\DeclareMathOperator{\Spec}{\mathrm{Spec}}
\DeclareMathOperator{\G}{\Gamma}
\newcommand{\mychi}{{\raise 2pt\hbox{$\chi$}}}
\definecolor{red2}{RGB}{160, 50, 50}
\definecolor{green2}{RGB}{50, 160, 50}
\definecolor{blue2}{RGB}{50, 50, 160}
\definecolor{magenta2}{RGB}{160, 48, 160}
\definecolor{cyan2}{RGB}{48, 160, 160}
\definecolor{yellow2}{RGB}{160, 160, 48}

%AMS-style theorem environments:

%\theoremstyle{plain} default

\newtheorem*{theorem*}{Theorem}

\newtheorem*{proposition*}{Proposition}

\theoremstyle{definition}
\newtheorem*{definition*}{Definition}
\newtheorem*{example*}{Example}
\newtheorem*{lemma*}{Lemma}
\newtheorem*{corollary*}{Corollary}
\newtheorem{exercise}{Exercise}
\newtheorem*{exercise*}{Exercise}

\theoremstyle{remark}
\newtheorem*{remark*}{Remark}
\newtheorem{hint}[theorem]{Hint}
\newtheorem*{hint*}{Hint}

\newtheorem*{notes*}{Notes}

%solution for custom numbering in theorems and lemmata:
\newtheorem{innercustomgeneric}{\customgenericname}
\providecommand{\customgenericname}{}
\newcommand{\newcustomtheorem}[2]{%
  \newenvironment{#1}[1]
  {%
   \renewcommand\customgenericname{#2}%
   \renewcommand\theinnercustomgeneric{##1}%
   \innercustomgeneric
  }
  {\endinnercustomgeneric}
}
\newcustomtheorem{customthm}{Theorem}
\newcustomtheorem{customlemma}{Lemma}
\newcustomtheorem{customcor}{Corollary}
\newcustomtheorem{customprop}{Proposition}
%%%%%%%%%%%%%%%% end of personal preamble %%%%%%%%%%%%%%%%
% ----------------------------------------------------------------
\begin{document}
\frontmatter      % Begin Roman style (i, ii, iii, iv...) page numbering

%
\UNIVERSITY{{THE UNIVERSITY OF MELBOURNE }}    
%
%%%%%%%%%%%%%%%%%%%%%%%%%%%%%%%%%%%%%%%%%%%%%%%%%%%%%%%%%%%%%%%%%%%%%%%%%
% Update your department and school here:
\department{{Faculty of Science}}
\school{{School of Mathematics and Statistics}}
%%%%%%%%%%%%%%%%%%%%%%%%%%%%%%%%%%%%%%%%%%%%%%%%%%%%%%%%%%%%%%%%%%%%%%%%%

%
%%%%%%%%%%%%%%%%%%%%%%%%%%%%%%%%%%%%%%%%%%%%%%%%%%%%%%%%%%%%%%%%%%%%%%%%%
% Set up the Title Page
% Change your thesis title and your information here
\title  {The Grothendieck-Teichm\"uller Group and the Operad of Parenthesized Braids}
\authors  {\texorpdfstring
            {\href{https://www.github.com/zin3724}{Yaxin Li}}
            {Yaxin Li}
            }
\addresses  {\groupname\\\deptname\\\univname}  % Do not change this here, instead these must be set in the "Thesis.cls" file, please look through it instead
\date       {\today}
\subject    {}
\keywords   {}
%%%%%%%%%%%%%%%%%%%%%%%%%%%%%%%%%%%%%%%%%%%%%%%%%%%%%%%%%%%%%%%%%%%%%%%%%

\maketitle
%% ----------------------------------------------------------------

% \setstretch{1.3}  % It is better to have smaller font and larger line spacing than the other way round

% Define the page headers using the FancyHdr package and set up for one-sided printing
\fancyhead{}  % Clears all page headers and footers
\rhead{\thepage}  % Sets the right side header to show the page number
\lhead{}  % Clears the left side page header

\pagestyle{fancy}  % Finally, use the "fancy" page style to implement the FancyHdr headers
\setlength{\headheight}{15pt}
\setstretch{1.0}
%% ----------------------------------------------------------------
% The Abstract Page

\addtotoc{Abstract}  % Add the "Abstract" page entry to the Contents
\abstract{
\addtocontents{toc}{\vspace{1em}}  % Add a gap in the Contents, for aesthetics

We introduce the operad of parenthesized braids $\PaB$, show that algebras over $\PaB$ correspond to braided monoidal categories, and describe the action of the Grothendieck-Teichm\"uller group $\GT$ on $\PaB$.
% (which comes from the $\GT$-action on Drinfeld associators)

}
\clearpage  % Abstract ended, start a new page

%% ----------------------------------------------------------------
% Declaration Page required for the Thesis, your institution may give you a different text to place here
% \input{Declaration.tex}
% \clearpage  % Declaration ended, now start a new page

%% ----------------------------------------------------------------
% Preface Page required for the Thesis, your institution may give you a different text to place here
% \input{Preface.tex}
% \clearpage  % Preface ended, now start a new page

%% ----------------------------------------------------------------
% % The Acknowledgements page, for thanking everyone
% % \setstretch{1.3}  % Reset the line-spacing to 1.3 for body text (if it has changed)
% \input{Acknowledgements.tex}
% \clearpage  % End of the Acknowledgements
% %% ----------------------------------------------------------------

\pagestyle{fancy}  %The page style headers have been "empty" all this time, now use the "fancy" headers as defined before to bring them back


%% ----------------------------------------------------------------
\lhead{\emph{Contents}}  % Set the left side page header to "Contents"
\tableofcontents  % Write out the Table of Contents

%% ----------------------------------------------------------------
% \lhead{\emph{List of Figures}}  % Set the left side page header to "List if Figures"
% \listoffigures  % Write out the List of Figures

%% ----------------------------------------------------------------
% \lhead{\emph{List of Tables}}  % Set the left side page header to "List of Tables"
% \listoftables  % Write out the List of Tables

%% ----------------------------------------------------------------
% % \setstretch{1.5}  % Set the line spacing to 1.5, this makes the following tables easier to read
% \clearpage  % Start a new page
% \lhead{\emph{Abbreviations}}  % Set the left side page header to "Abbreviations"
% \listofsymbols{ll}  % Include a list of Abbreviations (a table of two columns)
% {
% % \textbf{Acronym} & \textbf{W}hat (it) \textbf{S}tands \textbf{F}or \\
% \textbf{LAH} & \textbf{L}ist \textbf{A}bbreviations \textbf{H}ere \\

% }

% %% ----------------------------------------------------------------
% \clearpage  % Start a new page
% \lhead{\emph{Constants}}  % Set the left side page header to "Physical Constants"
% \listofconstants{lrcl}  % Include a list of Physical Constants (a four column table)
% {
% % Constant Name & Symbol & = & Constant Value (with units) \\
% Speed of Light & $c$ & $=$ & $2.997\ 924\ 58\times10^{8}\ \mbox{ms}^{-\mbox{s}}$ (exact)\\

% }

%% ----------------------------------------------------------------
% \clearpage  %Start a new page
% \lhead{\emph{Symbols}}  % Set the left side page header to "Symbols"

%   % Include a list of Symbols (a three column table)
% \listofnomenclature{lll}
% {
% % symbol & name & unit \\
% $a$ & distance & m \\
% $P$ & power & W (Js$^{-1}$) \\
% {}& & \\ % Gap to separate the Roman symbols from the Greek
% $\omega$ & angular frequency & rads$^{-1}$ \\
% }
% %% ----------------------------------------------------------------
% End of the pre-able, contents and lists of things


%% ----------------------------------------------------------------
\mainmatter	  % Begin normal, numeric (1,2,3...) page numbering
\pagestyle{fancy}  % Return the page headers back to the "fancy" style

% Include the chapters of the thesis, as separate files
% Just uncomment the lines as you write the chapters

\setstretch{1.0}  % Set the line spacing to 2.0 %%%%%%%%%%%%%%%%%%%%%%%%%%%%%%%%%%%%%%%%%%%%%%%%%%%%%%%%%%%%%%%%%%%%%%%%%%%%%%%%%%%%%
\chapter{Introduction}
\lhead{\emph{Introduction}}
 % The Backstory of PaB

%\chapter{\texorpdfstring{algebras of $\widehat{PaB}$}{algebras of P\^aB}}
\lhead{\emph{algebras of \widehat{PaB}}} % The action of GT^ on PaB

%\input{Chapter3} % 

%\input{Chapter4} % 

%\input{Chapter5} % 

%\input{Chapter6} % 

%\input{Chapter7} % 

%% ----------------------------------------------------------------
% Now begin the Appendices, including them as separate files

\addtocontents{toc}{\vspace{2em}} % Add a gap in the Contents, for aesthetics

\appendix % Cue to tell LaTeX that the following 'chapters' are Appendices

\input{AppendixA}	% Appendix Title

%\input{AppendixB} % Appendix Title

%\input{AppendixC} % Appendix Title

\addtocontents{toc}{\vspace{2em}}  % Add a gap in the Contents, for aesthetics
\backmatter

%% ----------------------------------------------------------------
\label{Bibliography}
\setstretch{1.0}
\lhead{\emph{Bibliography}}  % Change the left side page header to "Bibliography"
\bibliographystyle{unsrtnat}  % Use the "unsrtnat" BibTeX style for formatting the Bibliography
\bibliography{Bibliography}  % The references (bibliography) information are stored in the file named "Bibliography.bib"

\end{document}  % The End
%% ----------------------------------------------------------------