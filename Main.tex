\documentclass[a4paper, 12pt,openany,oneside]{amsbook}
\usepackage[square, numbers, comma, sort&compress]{natbib} 
\usepackage{verbatim}  % Needed for the "comment" environment to make LaTeX comments
\usepackage{vector}  % Allows "\bvec{}" and "\buvec{}" for "blackboard" style bold vectors in maths

%%%%%%%%%%%%%% my personal preamble %%%%%%%%%%%%%%%%
\usepackage[T1]{fontenc}
\usepackage[utf8]{inputenc}
\usepackage{graphicx,setspace,float,fancyhdr,hyperref,quiver,bbold,adjustbox}
\usepackage{amsmath,mathtools,amssymb,amsthm,cancel,mathdots,tikz-cd,xfrac,xcolor,amsfonts,esint}
\usepackage[all]{xy}
\usepackage[mathscr]{euscript}
% \usepackage[style=alphabetic]{biblatex}
\tikzcdset{scale cd/.style={every label/.append style={scale=#1},
    cells={nodes={scale=#1}}}}
% \setlength{\parindent}{0in}
\makeatletter
\DeclareMathOperator{\oeq}{\mathbin{\mathpalette\make@circled=}}%circled symbols
\DeclareMathOperator{\oneq}{\mathbin{\mathpalette\make@circled\neq}}
\newcommand{\make@circled}[2]{%
  \ooalign{$\m@th#1\smallbigcirc{#1}$\cr\hidewidth$\m@th#1#2$\hidewidth\cr}%
}
\newcommand{\smallbigcirc}[1]{%
  \vcenter{\hbox{\scalebox{0.77778}{$\m@th#1\bigcirc$}}}%
}
\makeatother
\renewcommand{\Re}{\mathrm{Re}}
\renewcommand{\Im}{\mathrm{Im}}
\newcommand{\defeq}{\overset{\mathrm{def}}{=}}
\newcommand{\bA}{\mathbb{A}}
\newcommand{\bC}{\mathbb{C}}
\newcommand{\bF}{\mathbb{F}}
\newcommand{\bN}{\mathbb{N}}
\newcommand{\bP}{\mathbb{P}}
\newcommand{\bQ}{\mathbb{Q}}
\newcommand{\bR}{\mathbb{R}}
\newcommand{\bZ}{\mathbb{Z}}
\newcommand{\lt}{\left}
\newcommand{\rt}{\right}
\newcommand{\rgl}{\rangle}
\newcommand{\lgl}{\langle}
\newcommand{\Cat}{\mathbf{Cat}}
\newcommand{\Fun}{\operatorname{Fun}}
\newcommand{\PaB}{\mathnormal{PaB}}
\newcommand{\PaBhat}{\widehat{\mathnormal{PaB}}}
\newcommand{\GT}{\mathsf{GT}}
\newcommand{\GThat}{\widehat{\mathsf{GT}}}
\newcommand{\symgrp}{\mathsf{S}}
\newcommand{\freegrp}{\mathsf{F}}
%change the style of arrows
\newcommand{\ra}{\longrightarrow}
\newcommand{\lra}{\longleftrightarrow}
\newcommand{\xra}{\xrightarrow}
\newcommand{\Ra}{\Rightarrow}
\newcommand{\Lra}{\Leftrightarrow}
\newcommand{\e}{\varepsilon}
\newcommand{\sC}{\mathscr{C}}
%\det, \dim, \ker, \lcm, \gcd, all trig functions, already built-in
\DeclareMathOperator{\id}{\mathrm{id}}
\DeclareMathOperator{\im}{\mathrm{im}}
\DeclareMathOperator{\tr}{\mathrm{tr}}
\DeclareMathOperator{\proj}{\mathrm{proj}}
\DeclareMathOperator{\End}{\mathrm{End}}
\DeclareMathOperator{\Aut}{\mathrm{Aut}}
\DeclareMathOperator{\Hom}{\mathrm{Hom}}
\DeclareMathOperator{\Ext}{\mathrm{Ext}}
\DeclareMathOperator{\Tor}{\mathrm{Tor}}
\DeclareMathOperator{\Span}{\mathrm{Span}}
\DeclareMathOperator{\Spec}{\mathrm{Spec}}
\DeclareMathOperator{\Obj}{\mathrm{Obj}}
\DeclareMathOperator{\G}{\Gamma}
\newcommand{\mychi}{{\raise 2pt\hbox{$\chi$}}}
\definecolor{red2}{RGB}{160, 50, 50}
\definecolor{green2}{RGB}{50, 160, 50}
\definecolor{blue2}{RGB}{50, 50, 160}
\definecolor{magenta2}{RGB}{160, 48, 160}
\definecolor{magentaMax}{RGB}{255, 0, 255}
\definecolor{cyan2}{RGB}{48, 160, 160}
\definecolor{cyanMax}{RGB}{0, 255, 255}
\definecolor{yellow2}{RGB}{160, 160, 48}
\definecolor{yellowMax}{RGB}{255, 255, 0}

%AMS-style theorem environments:

\theoremstyle{plain} %default

\newtheorem{theorem}{Theorem}
\newtheorem*{theorem*}{Theorem}

\newtheorem{proposition}[theorem]{Proposition}
\newtheorem*{proposition*}{Proposition}

\theoremstyle{definition}
\newtheorem{definition}[theorem]{Definition}
\newtheorem*{definition*}{Definition}
\newtheorem{example}[theorem]{Example}
\newtheorem*{example*}{Example}
\newtheorem{lemma}[theorem]{Lemma}
\newtheorem*{lemma*}{Lemma}
\newtheorem{corollary}[theorem]{Corollary}
\newtheorem*{corollary*}{Corollary}
\newtheorem{exercise}{Exercise}
\newtheorem*{exercise*}{Exercise}

\theoremstyle{remark}
\newtheorem{remark}[theorem]{Remark}
\newtheorem*{remark*}{Remark}
\newtheorem{hint}[theorem]{Hint}
\newtheorem*{hint*}{Hint}

\newtheorem*{notes*}{Notes}

%solution for custom numbering in theorems and lemmata:
\newtheorem{innercustomgeneric}{\customgenericname}
\providecommand{\customgenericname}{}
\newcommand{\newcustomtheorem}[2]{%
  \newenvironment{#1}[1]
  {%
   \renewcommand\customgenericname{#2}%
   \renewcommand\theinnercustomgeneric{##1}%
   \innercustomgeneric
  }
  {\endinnercustomgeneric}
}
\newcustomtheorem{customthm}{Theorem}
\newcustomtheorem{customlemma}{Lemma}
\newcustomtheorem{customcor}{Corollary}
\newcustomtheorem{customprop}{Proposition}
%%%%%%%%%%%%%%%% end of personal preamble %%%%%%%%%%%%%%%%
% ----------------------------------------------------------------
\begin{document}
\frontmatter      % Begin Roman style (i, ii, iii, iv...) page numbering
% Set up the Title Page
\title  {The Grothendieck-Teichm\"uller Group and the Operad of Parenthesized Braids}
\author{\texorpdfstring
            {\href{https://www.github.com/zin3724}{Yaxin Li}}{Yaxin Li}
        }
\date{\today}
\maketitle
%% ----------------------------------------------------------------

% \setstretch{1.3}  % It is better to have smaller font and larger line spacing than the other way round

% Define the page headers using the FancyHdr package and set up for one-sided printing
\fancyhead{}  % Clears all page headers and footers
\rhead{\thepage}  % Sets the right side header to show the page number
\lhead{}  % Clears the left side page header

\pagestyle{fancy}  % Finally, use the "fancy" page style to implement the FancyHdr headers
\setlength{\headheight}{15pt}
\setstretch{1.0}
%% ----------
\pagestyle{fancy}  %The page style headers have been "empty" all this time, now use the "fancy" headers as defined before to bring them back


%% ----------------------------------------------------------------
\lhead{\emph{Contents}}  % Set the left side page header to "Contents"
\tableofcontents  % Write out the Table of Contents

\mainmatter	  % Begin normal, numeric (1,2,3...) page numbering
\pagestyle{fancy}  % Return the page headers back to the "fancy" style

\setstretch{1.0}  % Set the line spacing to whatever
%%%%%%%%%%%%%%%%%%%%%%%%%%%%%%%%%%%%%%%%%%%%%%%%%%%%%%%%%%%%%%%%%%%%%%%%%%%%%%%%%%%%% the body %%%%%%%%%%%%%%%%%%%%%%%%%%%%%%%%%%%%%%%%%%%%%%%%%%%%%%%%%%%%%%%%%%%%%%%%%%%%%%%%%%%%%
% I'll leave it single spaced for now so that I can choose the right line spacing from 1.5 to 2.0 to make it into the desired page range
\chapter{Introduction}
\lhead{\emph{Introduction}}
 % The Backstory of PaB; it's a summary with citations of stuff to study later. Details will be in the appendix if I'm expected to write them up.

\chapter{\texorpdfstring{algebras of $\widehat{PaB}$}{algebras of P\^aB}}
\lhead{\emph{algebras of \widehat{PaB}}} % The action of GT^ on PaB

%\input{Chapter3} % 

%\input{Chapter4} % 

%\input{Chapter5} % 

%\input{Chapter6} % 

%\input{Chapter7} % 

%% ----------------------------------------------------------------
% Now begin the Appendices, including them as separate files

% \addtocontents{toc}{\vspace{2em}} % Add a gap in the Contents, for aesthetics

% \appendix % Cue to tell LaTeX that the following 'chapters' are Appendices

% \input{AppendixA}	% Appendix Title

% %\input{AppendixB} % Appendix Title

% %\input{AppendixC} % Appendix Title

\addtocontents{toc}{\vspace{2em}}  % Add a gap in the Contents, for aesthetics
\backmatter

%% ----------------------------------------------------------------
\label{Bibliography}
\setstretch{1.0}
\lhead{\emph{Bibliography}}  % Change the left side page header to "Bibliography"
\bibliographystyle{alpha}  % Use the "alpha" BibTeX style for formatting the Bibliography so that citations don't look like equivalence classes of numbers
\bibliography{Bibliography.bib}  % The references (bibliography) information are stored in the file named "Bibliography.bib"

\end{document}